% Preamble
% ---
\documentclass{article}

% Packages
% ---
\usepackage{amsmath} % Advanced math typesetting
\usepackage[utf8]{inputenc} % Unicode support (Umlauts etc.)
\usepackage[ngerman]{babel} % Change hyphenation rules
\usepackage{hyperref} % Add a link to your document
\usepackage{graphicx} % Add pictures to your document
\usepackage{listings} % Source code formatting and highlighting

\begin{document}
\section{Method}
This section describes the method used to detect the canal shoreline using both brightness constancy assumption and the dynamic motion of water.
\subsection{Features Tracking and Initialization}
A set amount of $k$ trackers is initially initialized using the Shi-Tomasi corner detection algorithm that detects good features to track. Due to the strong corners presented in the canal shorelines, several features will be initiated on the shoreline. Then a pyramidal implementation of Lucas-Kanade optical flow algorithm is used to displace the tracker across the subsequent $n$ frames. Due to the brightness constancy assumption of optical flow, trackers on water would be more likely to lost, leaving the remaining trackers on the canals.
\subsection{Entropy Calculation} 
The entropy calculations follows the one proposed by Pedro Santana. The equation of entropy is defined as:

\begin{align}
p \equiv (x_0,\ldots,x_n)\\
L = \sum_{i=0}^{n-1} \| x_i - x_{i-1} \| \\
H(p) = \frac{\log(\frac{L(p)}{d(p)})}{\log(n-1)}\cdot d_\theta (p)
\end{align}
Where $d(p)$ is the diameter of the minimum circle encompassing the trajectory; $d_\theta(p)$ is the scaled version of $d(p)$ so that the range of $d_\theta(\cdot)$ value across the set of trackers, $\{ d_\theta (x), \forall x \in \theta \}$ , is $[0,1]$; and $L(p)$ is defined as the length of the trajectory.

A given tracker $p$ is defined as a vector of $n$ positions relative to the t
\end{document}