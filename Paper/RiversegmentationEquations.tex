% Preamble
% ---
\documentclass{article}

% Packages
% ---
\usepackage{amsmath} % Advanced math typesetting
\usepackage[utf8]{inputenc} % Unicode support (Umlauts etc.)
\usepackage[ngerman]{babel} % Change hyphenation rules
\usepackage{hyperref} % Add a link to your document
\usepackage{graphicx} % Add pictures to your document
\usepackage{listings} % Source code formatting and highlighting

\begin{document}
\begin{align}
p \equiv (x_0,\ldots,x_n)\\
L = \sum_{i=0}^{n-1} \| x_i - x_{i-1} \| \\
H(p) = \frac{\log(\frac{L(p)}{d(p)})}{\log(n-1)}\cdot d_\theta (p)
\end{align}
Where $d(p)$ is the diameter of the minimum circle encompassing the trajectory; $d_\theta(p)$ is the scaled version of $d(p)$ so that the range of $d_\theta(\cdot)$ value across the set of trackers, $\{ d_\theta (x), \forall x \in \theta \}$ , is $[0,1]$; and $L(p)$ is defined as the length of the trajectory.

\begin{align}
a = (\theta , D)\\
d(p,q) = \sum_{i=0}^{n} \| a_{p,i} , a_{q,i} \| \\
D(p) = \frac{1}{\Phi (p)} \sum_{\forall q \in \Phi (p)} d (p,q)
\end{align}

where $D$ is the distance between the current and the previous point and $\theta$ is the absolute value of the angle between the line formed between the two points and the x-axis. $\Phi (p)$ represents the set of trackers composing the neighbourhood of $p$. The neighbours are the 4 immediate neighbours.

\begin{align}
F(p) = H_\theta(p) \cdot D_\theta(p)
\end{align}
The $H_\theta(p)$ and $D_\theta(p)$ are the normalized version of $H_(p)$ and $D_(p)$ respectively. The normalization procedure is the same as defined above.

\end{document}